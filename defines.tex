%%%%%%%%%%%%%%% COMPATIBILTY LAYER %%%%%%%%%%%%%

\providecommand\cvskip{}
\providecommand\cvtag[1]{-[#1]-}
\providecommand\divider\cvskip
\providecommand{\cvevent}[4]{\textbf{#1}\par#2\par#3\par#4}
\providecommand{\cvrow}[5]{\cvevent{#1}{#2}{#3}{#4}\par#5}

%%%%%%%%%%%%%%% INTERNALS %%%%%%%%%%%%%

\def\processitem#1{\cvtag{#1} } % do something with one item

\def\parsepair [[#1;;#2]]{%
\processitem{#1}%      % handle current item
\ifx\relax#2\relax
% no more items
\else
\parsepair [[#2]]%      % recurse on the rest
\fi
}

\def\simplelist\lbeg #1 \lend{\parsepair [[#1;;\relax]]}

\def\skillgrade#1{%
\ifx1#1 (di base)\else\ifx2#1 (discreta)\else\ifx3#1 (avanzata)\else\relax\fi\fi\fi
}


%%%%%%%%%%%%%%% OTHER COMMANDS %%%%%%%%%%%%%

\newcommand{\cvskills}[2]{%
\textbf{#1} \\[2mm]

{\raggedright\simplelist \lbeg #2 \lend}} % #1=title, #2=list

\newcommand{\cvdatefromto}[3][\relax]{#2 -- #3\ifx\relax#1\relax\else{ #1}\fi}
\newenvironment{citemize}{\begin{itemize}}{\end{itemize}}
\newcommand{\virg}[1]{``#1''}


%%%%%%%%%%%%%%% CONTENT COMMANDS %%%%%%%%%%%%%

%%%%%%%%%%%%%%% ITALIAN %%%%%%%%%%%%%

\newcommand{\myeducationIT}{% Istruzione
\cvrow{Laurea Magistrale in Scienze Statistiche}{Università degli Studi di Padova}{\cvdatefromto{Set. 2023}{Nov. 2025}}{}{Tesi: \textit{Network Control Risk Regression: An Integrated Approach to Network Meta-Analysis in Case-Control Studies}\\
Valutazione: 104/110
}

\divider

\cvrow{Laurea Triennale in Statistica Matematica e Trattamento Informatico dei Dati (SMID)}{Università degli Studi di Genova}{\cvdatefromto{Set. 2018}{Set. 2022}}{}{Tesi: \textit{Modelli demografici predittivi subregionali per la Liguria}\\
Valutazione: 109/110
}%END
}

\newcommand{\myworkIT}{% Esperienza lavorativa e di ricerca
	\cvrow{Analista dati e programmatore R/C++ (collaborazione coordinata e continuativa)}{Liguria Ricerche S.p.A.}{\cvdatefromto{Nov. 2024}{Lug. 2025}}{}{\begin{citemize}
		\item Riscrittura e significativo ampliamento del codice in {C++} redatto durante la Borsa di Ricerca.
		\item Aggiunta di funzionalità al modello demografico in termini di ripesatura e consistenza territoriale delle previsioni.
		\item Valutazione delle prestazioni del modello e produzione di una relazione finale sull'andamento della popolazione in Liguria.
		\item Organizzazione di {incontri di formazione} ai funzionari pubblici sull'utilizzo dello strumento.
\end{citemize}
}

\divider

\cvrow{Coordinatore della rilevazione \textit{Customer Satisfaction} sul turismo}{Camera di Commercio Genova}{\cvdatefromto{Lug. 2023}{Ott. 2023}}{}{\begin{citemize}%
		\item {Analisi dati} in tempo reale delle risposte ai questionari somministrati e {redazione di reportistica} sia generale che specifica per ciascun rilevatore.
		\item Costruzione della banca dati della rilevazione in {Excel} e utilizzo di {PowerQuery} per l'importazione dei dati.
		\item Supporto nella stesura del documento finale elaborato da Regione Liguria sull'analisi delle abitudini dei turisti.
\end{citemize}
}

\divider

\cvrow{Borsista di ricerca in previsioni demografiche per la Liguria al 2060}{Dipartimento di Scienze Politiche, Università degli Studi di Genova}{\cvdatefromto{Dic. 2022}{Ott. 2024}}{}{\begin{citemize}
		\item Sviluppo dell'algoritmo di previsione demografica a componenti di coorte, integrando modelli per la previsione delle singole componenti demografiche specifiche.
		\item Implementazione della metodologia Istat per la stima della popolazione a livello comunale e sub-comunale.
		\item Compilazione di un pacchetto ibrido R/C++.
		\item Sviluppo di una {web-app \emph{Shiny}} per interagire con il modello e i dati.
\end{citemize}
}

\divider

\cvrow{Tirocinio curriculare per lo sviluppo di modelli demografici previsionali per la Liguria}{Dipartimento di Matematica, Università degli Studi di Genova}{\cvdatefromto{Lug. 2022}{Ott. 2022}}{}{\begin{citemize}
		\item Creazione della {banca dati} di serie storiche contenente il bilancio della popolazione stratificato per sesso ed età, oltre a dati di natura geografica, reddituale ed occupazionale proveniente dal Censimento Permanente Popolazione e Abitazioni.
		\item Studio teorico approfondito dei modelli demografici predittivi, oggetto della tesi triennale.
		\item Sviluppo di un {modello demografico bayesiano} preliminare per la previsione demografica a livello provinciale.
		\item Analisi dei gruppi (\emph{cluster analysis}) in preparazione ad una futura aggregazione territoriale su base comunale.
\end{citemize}
}%END
}

\newcommand{\myeventsIT}{% Eventi
\cvrow{UNICARTradEconomy \& Finance}{Dipartimento di Scienze Aziendali e Giuridiche, Università degli Studi della Calabria}{\cvdatefromto[Mag. 2024]{7}{9}}{ Arcavacata di Rende (CS)}{%
	Intervento dal titolo \virg{\emph{Local Demographic Projections: Tackling Challenges of a Heterogeneous Territory}} per la presentazione del modello demografico regionale per la stima della popolazione sub-comunale.
}

\divider

\cvrow{Workshop demografia}{Aula Magna, Università degli Studi di Genova}{7 Giu. 2024}{}{%
	Intervento dal titolo \virg{\emph{Previsioni demografiche a livello locale per la Liguria: approcci modellistici in una realtà demografica eterogenea}} per la presentazione del modello demografico regionale per la stima della popolazione sub-comunale.

	Co-relatore dott. Gian Lorenzo Boracchia.
}

\divider

\cvrow{\emph{MILeS2023} | Milano - Impresa, Lavoro e Società}{Facoltà di Scienze Politiche, Economiche e Sociali, Università degli Studi di Milano}{4  Ott. 2023}{}{%
	Intervento dal titolo \virg{\emph{Previsioni demografiche per la Liguria: le sfide di un territorio eterogeneo}}, per la presentazione del modello demografico locale per la Liguria.

	Co-relatore dott. Mauro Natali.
}

\divider

\cvrow{\emph{StatCities} Olbia}{Isola di Peddona - Porto Vecchio s.n.c.}{\cvdatefromto[Giu. 2023]{15}{16}}{Olbia}{%
	Presentazione di due poster contenenti le principali attività svolte dalla Regione in ambito statistico, in particolare il modello di previsione demografica bayesiano e l'indagine \emph{Customer Satisfaction} sul turismo.
}%END
}



\newcommand{\myskillsIT}{% Competenze digitali
\cvskills{Programmazione statistica}{%
	R\skillgrade{3};;%
	Python\skillgrade{2};;%
	C++\skillgrade{2};;%
	SQL\skillgrade{2};;%
	Excel + PowerQuery\skillgrade{2};;%
	MATLAB\skillgrade{1};;%
	SAS\skillgrade{1};;%
	Julia\skillgrade{1}%
}

\divider

\cvskills{Programmazione \textit{front-end}}{%
	R/Shiny\skillgrade{3};;%
	PowerBI\skillgrade{1};;%
	HTML+CSS+JS\skillgrade{1}%
}

\divider

\cvskills{Impaginazione e grafica}{%
	RMarkdown/Quarto\skillgrade{3};;%
	LaTeX\skillgrade{2};;%
	Inkscape\skillgrade{1};;%
	MS Office/LibreOffice\skillgrade{3};;%
	Scribus\skillgrade{1};;%
	The GIMP\skillgrade{1};;%
	Typst\skillgrade{1}%
}

\divider

\cvskills{Sistemi operativi}{%
	Linux;;%
	Windows%
}%END
}

\newcommand{\myeventsIT}{% Eventi
\cvrow{UNICARTradEconomy \& Finance}{Dipartimento di Scienze Aziendali e Giuridiche, Università degli Studi della Calabria}{\cvdatefromto[Mag. 2024]{7}{9}}{ Arcavacata di Rende (CS)}{%
	Intervento dal titolo \virg{\emph{Local Demographic Projections: Tackling Challenges of a Heterogeneous Territory}} per la presentazione del modello demografico regionale per la stima della popolazione sub-comunale.
}

\divider

\cvrow{Workshop demografia}{Aula Magna, Università degli Studi di Genova}{7 Giu. 2024}{}{%
	Intervento dal titolo \virg{\emph{Previsioni demografiche a livello locale per la Liguria: approcci modellistici in una realtà demografica eterogenea}} per la presentazione del modello demografico regionale per la stima della popolazione sub-comunale.

	Co-relatore dott. Gian Lorenzo Boracchia.
}

\divider

\cvrow{\emph{MILeS2023} | Milano - Impresa, Lavoro e Società}{Facoltà di Scienze Politiche, Economiche e Sociali, Università degli Studi di Milano}{4  Ott. 2023}{}{%
	Intervento dal titolo \virg{\emph{Previsioni demografiche per la Liguria: le sfide di un territorio eterogeneo}}, per la presentazione del modello demografico locale per la Liguria.

	Co-relatore dott. Mauro Natali.
}

\divider

\cvrow{\emph{StatCities} Olbia}{Isola di Peddona - Porto Vecchio s.n.c.}{\cvdatefromto[Giu. 2023]{15}{16}}{Olbia}{%
	Presentazione di due poster contenenti le principali attività svolte dalla Regione in ambito statistico, in particolare il modello di previsione demografica bayesiano e l'indagine \emph{Customer Satisfaction} sul turismo.
}%END
}



%%%%%%%%%%%%%%%% ENGLISH %%%%%%%%%%%%%%%%%%

\newcommand{\myeducationEN}{% Education
\cvrow{Master's Degree in Statistical Sciences}{University of Padova}{\cvdatefromto{Sep. 2023}{Nov. 2025}}{}{Thesis: \textit{Network Control Risk Regression: An Integrated Approach to Network Meta-Analysis in Case-Control Studies}\\
Grade: 104/110
}

\divider

\cvrow{Bachelor's Degree in Mathematical Statistics and Data Processing (SMID)}{University of Genova}{\cvdatefromto{Sep. 2018}{Sep. 2022}}{}{Thesis: \textit{Predictive subregional demographic models for the Liguria Region}\\
Grade: 109/110
}%END
}

\newcommand{\myworkEN}{% Research and work experience
\cvrow{Data analyst and R/C++ programmer (coordinated and continuous collaboration)}{Liguria Ricerche S.p.A.}{\cvdatefromto{Nov. 2024}{Jul. 2025}}{}{\begin{citemize}
\item Rewriting and significant expansion of the C++ code developed during the Research Fellowship.
\item Added functionalities to the demographic model in terms of reweighting and territorial consistency of forecasts.
\item Performance evaluation of the model and production of a final report on population trends in Liguria.
\item Organization of training sessions for public officials on the use of the software.
\end{citemize}
}

\divider

\cvrow{Coordinator of the \textit{Customer Satisfaction} survey on tourism}{Chamber of Commerce of Genova}{\cvdatefromto{Jul. 2023}{Oct. 2023}}{}{\begin{citemize}
\item Real-time data analysis of questionnaire responses and drafting of reports both general and specific for each surveyor.
\item Construction of the survey database in Excel and use of PowerQuery for data import.
\item Support in drafting the final document prepared by Liguria Region on the analysis of tourist habits.
\end{citemize}
}

\divider

\cvrow{Research Fellow in demographic forecasting for Liguria to 2060}{Department of Political Sciences, University of Genova}{\cvdatefromto{Dec. 2022}{Oct. 2024}}{}{\begin{citemize}
\item Development of the cohort-component demographic forecast algorithm, integrating models for the prediction of specific demographic components.
\item Implementation of the Istat methodology for population estimation at municipal and sub-municipal levels.
\item Compilation of a hybrid R/C++ package.
\item Development of a Shiny web-app to interact with the model and data.
\end{citemize}
}

\divider

\cvrow{Curricular internship for developing predictive demographic models for Liguria}{Department of Mathematics, University of Genova}{\cvdatefromto{Jul. 2022}{Oct. 2022}}{}{\begin{citemize}
\item Creation of the database of time series containing the population balance stratified by sex and age, as well as geographical, income, and employment data from the Permanent Population and Housing Census.
\item In-depth theoretical study of predictive demographic models, subject of the bachelor's thesis.
\item Development of a preliminary Bayesian demographic model for demographic forecasting at the provincial level.
\item Cluster analysis in preparation for future territorial aggregation on a municipal basis.
\end{citemize}
}%END
}

\newcommand{\myskillsEN}{% Digital skills
\cvskills{Statistical programming}{%
	R\skillgrade{3};;%
	Python\skillgrade{2};;%
	C++\skillgrade{2};;%
	SQL\skillgrade{2};;%
	Excel + PowerQuery\skillgrade{2};;%
	MATLAB\skillgrade{1};;%
	SAS\skillgrade{1};;%
	Julia\skillgrade{1}%
}

\divider

\cvskills{Frontend development}{%
	R/Shiny\skillgrade{3};;%
	PowerBI\skillgrade{1};;%
	HTML+CSS+JS\skillgrade{1}%
}

\divider

\cvskills{Authoring and graphics}{%
	RMarkdown/Quarto\skillgrade{3};;%
	LaTeX\skillgrade{2};;%
	Inkscape\skillgrade{1};;%
	MS Office/LibreOffice\skillgrade{3};;%
	Scribus\skillgrade{1};;%
	The GIMP\skillgrade{1};;%
	Typst\skillgrade{1}%
}

\divider

\cvskills{Operating systems}{%
	Linux;;%
	Windows%
}%END
}


\newcommand{\myeventsEN}{% Events
\cvrow{UNICARTradEconomy \& Finance}{Department of Business and Legal Sciences, University of Calabria}{\cvdatefromto[May 2024]{7}{9}}{Arcavacata di Rende (CS)}{%
Presentation titled \virg{Local Demographic Projections: Tackling Challenges of a Heterogeneous Territory} for the presentation of the regional demographic model for sub-municipal population estimation.
}

\divider

\cvrow{Demography Workshop}{Aula Magna, University of Genova}{7 Jun. 2024}{}{%
Presentation titled \virg{Local Demographic Projections for Liguria: Modeling Approaches in a Heterogeneous Demographic Reality} for the presentation of the regional demographic model for sub-municipal population estimation.

Co-speaker Dr. Gian Lorenzo Boracchia.
}

\divider

\cvrow{\emph{MILeS2023} | Milano - Impresa, Lavoro e Società}{Faculty of Political, Economic and Social Sciences, University of Milano}{4 Oct. 2023}{}{%
Presentation titled \virg{Demographic Projections for Liguria: Challenges of a Heterogeneous Territory}, for the presentation of the local demographic model for Liguria.

Co-speaker Dr. Mauro Natali.
}

\divider

\cvrow{\emph{StatCities} Olbia}{Isola di Peddona - Porto Vecchio s.n.c.}{\cvdatefromto[Jun. 2023]{15}{16}}{Olbia}{%
Presentation of two posters containing the main activities carried out by the Region in the statistical field, in particular the Bayesian demographic projection model and the Customer Satisfaction survey on tourism.
}%END
}
