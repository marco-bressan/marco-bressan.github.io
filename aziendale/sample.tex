%%%%%%%%%%%%%%%%%
% This is an sample CV template created using altacv.cls
% (v1.7.2, 28 August 2024) written by LianTze Lim (liantze@gmail.com). Compiles with pdfLaTeX, XeLaTeX and LuaLaTeX.
%
%% It may be distributed and/or modified under the
%% conditions of the LaTeX Project Public License, either version 1.3
%% of this license or (at your option) any later version.
%% The latest version of this license is in
%%    http://www.latex-project.org/lppl.txt
%% and version 1.3 or later is part of all distributions of LaTeX
%% version 2003/12/01 or later.
%%%%%%%%%%%%%%%%

%% Use the "normalphoto" option if you want a normal photo instead of cropped to a circle
% \documentclass[10pt,a4paper,normalphoto]{altacv}

\documentclass[10pt,a4paper,ragged2e,withhyper]{altacv}
%% AltaCV uses the fontawesome5 and simpleicons packages.
%% See http://texdoc.net/pkg/fontawesome5 and http://texdoc.net/pkg/simpleicons for full list of symbols.
\usepackage[italian]{babel}
% Change the page layout if you need to
\geometry{left=1.25cm,right=1.25cm,top=1.5cm,bottom=1.5cm,columnsep=1.2cm}

% The paracol package lets you typeset columns of text in parallel
\usepackage{paracol}

% Change the font if you want to, depending on whether
% you're using pdflatex or xelatex/lualatex
% WHEN COMPILING WITH XELATEX PLEASE USE
% xelatex -shell-escape -output-driver="xdvipdfmx -z 0" sample.tex
\iftutex 
  % If using xelatex or lualatex:
  \setmainfont{Source Sans Pro}
  \setsansfont{Arial}
  \renewcommand{\familydefault}{\sfdefault}
\else
  % If using pdflatex:
  \usepackage[rm]{roboto}
  \usepackage[defaultsans]{lato}
  % \usepackage{sourcesanspro}
  \renewcommand{\familydefault}{\sfdefault}
\fi

% Change the colours if you want to
\definecolor{SlateGrey}{HTML}{2E2E2E}
\definecolor{LightGrey}{HTML}{666671}
\definecolor{DarkTitleColor}{HTML}{550C18}
\definecolor{TitleColor}{HTML}{992243}
\definecolor{GoldenEarth}{HTML}{D5C9C3}
\colorlet{name}{DarkTitleColor}
\colorlet{tagline}{TitleColor}
\colorlet{heading}{DarkTitleColor}
\colorlet{headingrule}{GoldenEarth}
\colorlet{subheading}{TitleColor}
\colorlet{accent}{TitleColor}
\colorlet{emphasis}{SlateGrey}
\colorlet{body}{LightGrey}

% Change some fonts, if necessary
\renewcommand{\namefont}{\Huge\rmfamily\bfseries}
\renewcommand{\personalinfofont}{\footnotesize}
\renewcommand{\cvsectionfont}{\LARGE\rmfamily\bfseries}
\renewcommand{\cvsubsectionfont}{\large\bfseries}


% Change the bullets for itemize and rating marker
% for \cvskill if you want to
\renewcommand{\cvItemMarker}{{\small\textbullet}}
\renewcommand{\cvRatingMarker}{\faCircle}
% ...and the markers for the date/location for \cvevent
% \renewcommand{\cvDateMarker}{\faCalendar*[regular]}
% \renewcommand{\cvLocationMarker}{\faMapMarker*}

%%%%%%%%%%%%%%% COMPATIBILTY LAYER %%%%%%%%%%%%%

\providecommand\cvskip{}
\providecommand\cvtag[1]{-[#1]-}
\providecommand\divider\cvskip
\providecommand{\cvevent}[4]{\textbf{#1}\par#2\par#3\par#4}
\providecommand{\cvrow}[5]{\cvevent{#1}{#2}{#3}{#4}\par#5}

%%%%%%%%%%%%%%% INTERNALS %%%%%%%%%%%%%

\def\processitem#1{\cvtag{#1} } % do something with one item

\def\parsepair [[#1;;#2]]{%
\processitem{#1}%      % handle current item
\ifx\relax#2\relax
% no more items
\else
\parsepair [[#2]]%      % recurse on the rest
\fi
}

\def\simplelist\lbeg #1 \lend{\parsepair [[#1;;\relax]]}

\def\skillgrade#1{%
\ifx1#1 (di base)\else\ifx2#1 (discreta)\else\ifx3#1 (avanzata)\else\relax\fi\fi\fi
}


%%%%%%%%%%%%%%% OTHER COMMANDS %%%%%%%%%%%%%

\newcommand{\cvskills}[2]{%
\textbf{#1} \\[2mm]

{\raggedright\simplelist \lbeg #2 \lend}} % #1=title, #2=list

\newcommand{\cvdatefromto}[3][\relax]{#2 -- #3\ifx\relax#1\relax\else{ #1}\fi}
\newenvironment{citemize}{\begin{itemize}}{\end{itemize}}
\newcommand{\virg}[1]{``#1''}


%%%%%%%%%%%%%%% CONTENT COMMANDS %%%%%%%%%%%%%

%%%%%%%%%%%%%%% ITALIAN %%%%%%%%%%%%%

\newcommand{\myeducationIT}{% Istruzione
\cvrow{Laurea Magistrale in Scienze Statistiche}{Università degli Studi di Padova}{\cvdatefromto{Set. 2023}{Nov. 2025}}{}{Tesi: \textit{Network Control Risk Regression: An Integrated Approach to Network Meta-Analysis in Case-Control Studies}\\
Valutazione: 104/110
}

\divider

\cvrow{Laurea Triennale in Statistica Matematica e Trattamento Informatico dei Dati (SMID)}{Università degli Studi di Genova}{\cvdatefromto{Set. 2018}{Set. 2022}}{}{Tesi: \textit{Modelli demografici predittivi subregionali per la Liguria}\\
Valutazione: 109/110
}%END
}

\newcommand{\myworkIT}{% Esperienza lavorativa e di ricerca
	\cvrow{Analista dati e programmatore R/C++ (collaborazione coordinata e continuativa)}{Liguria Ricerche S.p.A.}{\cvdatefromto{Nov. 2024}{Lug. 2025}}{}{\begin{citemize}
		\item Riscrittura e significativo ampliamento del codice in {C++} redatto durante la Borsa di Ricerca.
		\item Aggiunta di funzionalità al modello demografico in termini di ripesatura e consistenza territoriale delle previsioni.
		\item Valutazione delle prestazioni del modello e produzione di una relazione finale sull'andamento della popolazione in Liguria.
		\item Organizzazione di {incontri di formazione} ai funzionari pubblici sull'utilizzo dello strumento.
\end{citemize}
}

\divider

\cvrow{Coordinatore della rilevazione \textit{Customer Satisfaction} sul turismo}{Camera di Commercio Genova}{\cvdatefromto{Lug. 2023}{Ott. 2023}}{}{\begin{citemize}%
		\item {Analisi dati} in tempo reale delle risposte ai questionari somministrati e {redazione di reportistica} sia generale che specifica per ciascun rilevatore.
		\item Costruzione della banca dati della rilevazione in {Excel} e utilizzo di {PowerQuery} per l'importazione dei dati.
		\item Supporto nella stesura del documento finale elaborato da Regione Liguria sull'analisi delle abitudini dei turisti.
\end{citemize}
}

\divider

\cvrow{Borsista di ricerca in previsioni demografiche per la Liguria al 2060}{Dipartimento di Scienze Politiche, Università degli Studi di Genova}{\cvdatefromto{Dic. 2022}{Ott. 2024}}{}{\begin{citemize}
		\item Sviluppo dell'algoritmo di previsione demografica a componenti di coorte, integrando modelli per la previsione delle singole componenti demografiche specifiche.
		\item Implementazione della metodologia Istat per la stima della popolazione a livello comunale e sub-comunale.
		\item Compilazione di un pacchetto ibrido R/C++.
		\item Sviluppo di una {web-app \emph{Shiny}} per interagire con il modello e i dati.
\end{citemize}
}

\divider

\cvrow{Tirocinio curriculare per lo sviluppo di modelli demografici previsionali per la Liguria}{Dipartimento di Matematica, Università degli Studi di Genova}{\cvdatefromto{Lug. 2022}{Ott. 2022}}{}{\begin{citemize}
		\item Creazione della {banca dati} di serie storiche contenente il bilancio della popolazione stratificato per sesso ed età, oltre a dati di natura geografica, reddituale ed occupazionale proveniente dal Censimento Permanente Popolazione e Abitazioni.
		\item Studio teorico approfondito dei modelli demografici predittivi, oggetto della tesi triennale.
		\item Sviluppo di un {modello demografico bayesiano} preliminare per la previsione demografica a livello provinciale.
		\item Analisi dei gruppi (\emph{cluster analysis}) in preparazione ad una futura aggregazione territoriale su base comunale.
\end{citemize}
}%END
}

\newcommand{\myeventsIT}{% Eventi
\cvrow{UNICARTradEconomy \& Finance}{Dipartimento di Scienze Aziendali e Giuridiche, Università degli Studi della Calabria}{\cvdatefromto[Mag. 2024]{7}{9}}{ Arcavacata di Rende (CS)}{%
	Intervento dal titolo \virg{\emph{Local Demographic Projections: Tackling Challenges of a Heterogeneous Territory}} per la presentazione del modello demografico regionale per la stima della popolazione sub-comunale.
}

\divider

\cvrow{Workshop demografia}{Aula Magna, Università degli Studi di Genova}{7 Giu. 2024}{}{%
	Intervento dal titolo \virg{\emph{Previsioni demografiche a livello locale per la Liguria: approcci modellistici in una realtà demografica eterogenea}} per la presentazione del modello demografico regionale per la stima della popolazione sub-comunale.

	Co-relatore dott. Gian Lorenzo Boracchia.
}

\divider

\cvrow{\emph{MILeS2023} | Milano - Impresa, Lavoro e Società}{Facoltà di Scienze Politiche, Economiche e Sociali, Università degli Studi di Milano}{4  Ott. 2023}{}{%
	Intervento dal titolo \virg{\emph{Previsioni demografiche per la Liguria: le sfide di un territorio eterogeneo}}, per la presentazione del modello demografico locale per la Liguria.

	Co-relatore dott. Mauro Natali.
}

\divider

\cvrow{\emph{StatCities} Olbia}{Isola di Peddona - Porto Vecchio s.n.c.}{\cvdatefromto[Giu. 2023]{15}{16}}{Olbia}{%
	Presentazione di due poster contenenti le principali attività svolte dalla Regione in ambito statistico, in particolare il modello di previsione demografica bayesiano e l'indagine \emph{Customer Satisfaction} sul turismo.
}%END
}



\newcommand{\myskillsIT}{% Competenze digitali
\cvskills{Programmazione statistica}{%
	R\skillgrade{3};;%
	Python\skillgrade{2};;%
	C++\skillgrade{2};;%
	SQL\skillgrade{2};;%
	Excel + PowerQuery\skillgrade{2};;%
	MATLAB\skillgrade{1};;%
	SAS\skillgrade{1};;%
	Julia\skillgrade{1}%
}

\divider

\cvskills{Programmazione \textit{front-end}}{%
	R/Shiny\skillgrade{3};;%
	PowerBI\skillgrade{1};;%
	HTML+CSS+JS\skillgrade{1}%
}

\divider

\cvskills{Impaginazione e grafica}{%
	RMarkdown/Quarto\skillgrade{3};;%
	LaTeX\skillgrade{2};;%
	Inkscape\skillgrade{1};;%
	MS Office/LibreOffice\skillgrade{3};;%
	Scribus\skillgrade{1};;%
	The GIMP\skillgrade{1};;%
	Typst\skillgrade{1}%
}

\divider

\cvskills{Sistemi operativi}{%
	Linux;;%
	Windows%
}%END
}

\newcommand{\myeventsIT}{% Eventi
\cvrow{UNICARTradEconomy \& Finance}{Dipartimento di Scienze Aziendali e Giuridiche, Università degli Studi della Calabria}{\cvdatefromto[Mag. 2024]{7}{9}}{ Arcavacata di Rende (CS)}{%
	Intervento dal titolo \virg{\emph{Local Demographic Projections: Tackling Challenges of a Heterogeneous Territory}} per la presentazione del modello demografico regionale per la stima della popolazione sub-comunale.
}

\divider

\cvrow{Workshop demografia}{Aula Magna, Università degli Studi di Genova}{7 Giu. 2024}{}{%
	Intervento dal titolo \virg{\emph{Previsioni demografiche a livello locale per la Liguria: approcci modellistici in una realtà demografica eterogenea}} per la presentazione del modello demografico regionale per la stima della popolazione sub-comunale.

	Co-relatore dott. Gian Lorenzo Boracchia.
}

\divider

\cvrow{\emph{MILeS2023} | Milano - Impresa, Lavoro e Società}{Facoltà di Scienze Politiche, Economiche e Sociali, Università degli Studi di Milano}{4  Ott. 2023}{}{%
	Intervento dal titolo \virg{\emph{Previsioni demografiche per la Liguria: le sfide di un territorio eterogeneo}}, per la presentazione del modello demografico locale per la Liguria.

	Co-relatore dott. Mauro Natali.
}

\divider

\cvrow{\emph{StatCities} Olbia}{Isola di Peddona - Porto Vecchio s.n.c.}{\cvdatefromto[Giu. 2023]{15}{16}}{Olbia}{%
	Presentazione di due poster contenenti le principali attività svolte dalla Regione in ambito statistico, in particolare il modello di previsione demografica bayesiano e l'indagine \emph{Customer Satisfaction} sul turismo.
}%END
}



%%%%%%%%%%%%%%%% ENGLISH %%%%%%%%%%%%%%%%%%

\newcommand{\myeducationEN}{% Education
\cvrow{Master's Degree in Statistical Sciences}{University of Padova}{\cvdatefromto{Sep. 2023}{Nov. 2025}}{}{Thesis: \textit{Network Control Risk Regression: An Integrated Approach to Network Meta-Analysis in Case-Control Studies}\\
Grade: 104/110
}

\divider

\cvrow{Bachelor's Degree in Mathematical Statistics and Data Processing (SMID)}{University of Genova}{\cvdatefromto{Sep. 2018}{Sep. 2022}}{}{Thesis: \textit{Predictive subregional demographic models for the Liguria Region}\\
Grade: 109/110
}%END
}

\newcommand{\myworkEN}{% Research and work experience
\cvrow{Data analyst and R/C++ programmer (coordinated and continuous collaboration)}{Liguria Ricerche S.p.A.}{\cvdatefromto{Nov. 2024}{Jul. 2025}}{}{\begin{citemize}
\item Rewriting and significant expansion of the C++ code developed during the Research Fellowship.
\item Added functionalities to the demographic model in terms of reweighting and territorial consistency of forecasts.
\item Performance evaluation of the model and production of a final report on population trends in Liguria.
\item Organization of training sessions for public officials on the use of the software.
\end{citemize}
}

\divider

\cvrow{Coordinator of the \textit{Customer Satisfaction} survey on tourism}{Chamber of Commerce of Genova}{\cvdatefromto{Jul. 2023}{Oct. 2023}}{}{\begin{citemize}
\item Real-time data analysis of questionnaire responses and drafting of reports both general and specific for each surveyor.
\item Construction of the survey database in Excel and use of PowerQuery for data import.
\item Support in drafting the final document prepared by Liguria Region on the analysis of tourist habits.
\end{citemize}
}

\divider

\cvrow{Research Fellow in demographic forecasting for Liguria to 2060}{Department of Political Sciences, University of Genova}{\cvdatefromto{Dec. 2022}{Oct. 2024}}{}{\begin{citemize}
\item Development of the cohort-component demographic forecast algorithm, integrating models for the prediction of specific demographic components.
\item Implementation of the Istat methodology for population estimation at municipal and sub-municipal levels.
\item Compilation of a hybrid R/C++ package.
\item Development of a Shiny web-app to interact with the model and data.
\end{citemize}
}

\divider

\cvrow{Curricular internship for developing predictive demographic models for Liguria}{Department of Mathematics, University of Genova}{\cvdatefromto{Jul. 2022}{Oct. 2022}}{}{\begin{citemize}
\item Creation of the database of time series containing the population balance stratified by sex and age, as well as geographical, income, and employment data from the Permanent Population and Housing Census.
\item In-depth theoretical study of predictive demographic models, subject of the bachelor's thesis.
\item Development of a preliminary Bayesian demographic model for demographic forecasting at the provincial level.
\item Cluster analysis in preparation for future territorial aggregation on a municipal basis.
\end{citemize}
}%END
}

\newcommand{\myskillsEN}{% Digital skills
\cvskills{Statistical programming}{%
	R\skillgrade{3};;%
	Python\skillgrade{2};;%
	C++\skillgrade{2};;%
	SQL\skillgrade{2};;%
	Excel + PowerQuery\skillgrade{2};;%
	MATLAB\skillgrade{1};;%
	SAS\skillgrade{1};;%
	Julia\skillgrade{1}%
}

\divider

\cvskills{Frontend development}{%
	R/Shiny\skillgrade{3};;%
	PowerBI\skillgrade{1};;%
	HTML+CSS+JS\skillgrade{1}%
}

\divider

\cvskills{Authoring and graphics}{%
	RMarkdown/Quarto\skillgrade{3};;%
	LaTeX\skillgrade{2};;%
	Inkscape\skillgrade{1};;%
	MS Office/LibreOffice\skillgrade{3};;%
	Scribus\skillgrade{1};;%
	The GIMP\skillgrade{1};;%
	Typst\skillgrade{1}%
}

\divider

\cvskills{Operating systems}{%
	Linux;;%
	Windows%
}%END
}


\newcommand{\myeventsEN}{% Events
\cvrow{UNICARTradEconomy \& Finance}{Department of Business and Legal Sciences, University of Calabria}{\cvdatefromto[May 2024]{7}{9}}{Arcavacata di Rende (CS)}{%
Presentation titled \virg{Local Demographic Projections: Tackling Challenges of a Heterogeneous Territory} for the presentation of the regional demographic model for sub-municipal population estimation.
}

\divider

\cvrow{Demography Workshop}{Aula Magna, University of Genova}{7 Jun. 2024}{}{%
Presentation titled \virg{Local Demographic Projections for Liguria: Modeling Approaches in a Heterogeneous Demographic Reality} for the presentation of the regional demographic model for sub-municipal population estimation.

Co-speaker Dr. Gian Lorenzo Boracchia.
}

\divider

\cvrow{\emph{MILeS2023} | Milano - Impresa, Lavoro e Società}{Faculty of Political, Economic and Social Sciences, University of Milano}{4 Oct. 2023}{}{%
Presentation titled \virg{Demographic Projections for Liguria: Challenges of a Heterogeneous Territory}, for the presentation of the local demographic model for Liguria.

Co-speaker Dr. Mauro Natali.
}

\divider

\cvrow{\emph{StatCities} Olbia}{Isola di Peddona - Porto Vecchio s.n.c.}{\cvdatefromto[Jun. 2023]{15}{16}}{Olbia}{%
Presentation of two posters containing the main activities carried out by the Region in the statistical field, in particular the Bayesian demographic projection model and the Customer Satisfaction survey on tourism.
}%END
}

\renewcommand{\skillgrade}[1]{\small{%
\ifx1#1 
 \faIcon{star}\faIcon[regular]{star}\faIcon[regular]{star}
\else\ifx2#1
 \faIcon{star}\faIcon{star}\faIcon[regular]{star}
\else\ifx3#1
 \faIcon{star}\faIcon{star}\faIcon{star}
\else\relax
\fi
\fi
\fi
}}



% If your CV/résumé is in a language other than English,
% then you probably want to change these so that when you
% copy-paste from the PDF or run pdftotext, the location
% and date marker icons for \cvevent will paste as correct
% translations. For example Spanish:
% \renewcommand{\locationname}{Ubicación}
% \renewcommand{\datename}{Fecha}


%% Use (and optionally edit if necessary) this .tex if you
%% want to use an author-year reference style like APA(6)
%% for your publication list
 \input{pubs-authoryear.tex}

%% Use (and optionally edit if necessary) this .tex if you
%% want an originally numerical reference style like IEEE
%% for your publication list
% \input{pubs-num.tex}

%% sample.bib contains your publications
\addbibresource{../cv.bib}
% \usepackage{academicons}\let\faOrcid\aiOrcid
\begin{document}
\name{Marco Bressan}
\tagline{Ricercatore statistico e analista dati}
%% You can add multiple photos on the left or right
\photoR{2.8cm}{foto_cv}
% \photoL{2.5cm}{Yacht_High,Suitcase_High}

\personalinfo{%
  % Not all of these are required!
  \email{marco.bressan00@outlook.it}
  \phone{(+39) 340 - 071 16 76}
  \mailaddress{Via Foscolo 1/7, 17047 Quiliano (SV)}
  \homepage{marco-bressan.github.io}
  % \twitter{@twitterhandle}
  \printinfo{\faLinkedin}{Marco Bressan}[https://www.linkedin.com/in/marco-bressan-8885b6252/]
  \github{marco-bressan}
  \orcid{0009-0009-8265-326X}
  %% You can add your own arbitrary detail with
  %% \printinfo{symbol}{detail}[optional hyperlink prefix]
  % \printinfo{\faPaw}{Hey ho!}[https://example.com/]

  %% Or you can declare your own field with
  %% \NewInfoFiled{fieldname}{symbol}[optional hyperlink prefix] and use it:
  % \NewInfoField{gitlab}{\faGitlab}[https://gitlab.com/]
  % \gitlab{your_id}
  %%
  %% For services and platforms like Mastodon where there isn't a
  %% straightforward relation between the user ID/nickname and the hyperlink,
  %% you can use \printinfo directly e.g.
  % \printinfo{\faMastodon}{@username@instace}[https://instance.url/@username]
  %% But if you absolutely want to create new dedicated info fields for
  %% such platforms, then use \NewInfoField* with a star:
  % \NewInfoField*{mastodon}{\faMastodon}
  %% then you can use \mastodon, with TWO arguments where the 2nd argument is
  %% the full hyperlink.
  % \mastodon{@username@instance}{https://instance.url/@username}
}

\makecvheader
%% Depending on your tastes, you may want to make fonts of itemize environments slightly smaller
% \AtBeginEnvironment{itemize}{\small}

%% Set the left/right column width ratio to 6:4.
\columnratio{0.6}

% Start a 2-column paracol. Both the left and right columns will automatically
% break across pages if things get too long.
\begin{paracol}{2}


\cvsection{Esperienza lavorativa}

\myworkIT

% \cvevent{Stage come operatore al front-desk}{Ufficio IAT di Finale Ligure}{05/2017 -- 06/2017}{Regione Liguria, Genova}
% \begin{itemize}
% \item Attività di \textbf{assistenza e informazione} ai turisti, sia in lingua italiana, sia in \textbf{inglese}. 
% \item Promozione delle eccellenze naturali, storiche e culturali del Finalese.
% \item Cura della sinergia con gli alberghi, al fine di massimizzare la ricettività del territorio intercettando i turisti di passaggio.
% \end{itemize}

\medskip


% \cvsection{A Day of My Life}

% % Adapted from @Jake's answer from http://tex.stackexchange.com/a/82729/226
% % \wheelchart{outer radius}{inner radius}{
% % comma-separated list of value/text width/color/detail}
% \wheelchart{1.5cm}{0.5cm}{%
%   6/8em/accent!30/{Sleep,\\beautiful sleep},
%   3/8em/accent!40/Hopeful novelist by night,
%   8/8em/accent!60/Daytime job,
%   2/10em/accent/Sports and relaxation,
%   5/6em/accent!20/Spending time with family
% }

% % use ONLY \newpage if you want to force a page break for
% % ONLY the current column
% \newpage

% \cvsection{Competenze trasversali e relazionali} 

% \textbf{\large Curiosità ed entusiasmo per l'apprendimento} \smallskip
% \begin{itemize}
%     \item L'idea di \textbf{imparare costantemente} qualcosa di nuovo è per me estremamente gratificante, nonchè uno degli aspetti cardine del mio lavoro ideale.
%     \item Apprezzo la sfida e l'esser messo in discussione.
% \end{itemize}

% \divider

% \textbf{\large Capacità espositive} \smallskip
% \begin{itemize}
%     \item L'esperienza lavorativa pregressa mi ha dato la possibilità di tenere interventi in eventi congressuali a tema statistico, migliorando la mia \textbf{capacità oratoria} in pubblico, in particolare indirizzata ad un \textbf{pubblico eterogeneo}, anche non tecnico.
%     \item Ho avuto modo di redigere reportistica indirizzata a figure politiche, per le quali capacità di sintesi e orientamento al risultato sono aspetti imprescindibili.
%     \item Ho sviluppato spiccate abilità nella produzione di \textbf{presentazioni accattivanti e concise}, sia in PowerPoint che in LaTeX.
% \end{itemize}

% \divider

% \textbf{\large Lavoro di squadra} \smallskip
% \begin{itemize}
%     \item Lavorando a stretto giro con colleghi di formazione eterogenea, ho imparato a relazionarmi con ciascuno di loro in modo efficace.
%     \item Ho ricoperto un ruolo di coordinamento in un piccolo progetto universitario, sperimentando l'\textbf{assegnazione dei compiti} all'interno del gruppo, l'\textbf{ascolto attivo} delle varie proposte e la presa di \textbf{decisioni} per giungere al risultato.
% \end{itemize}

% \medskip

%\divider

% \textbf{\large Altre cose inutili} \smallskip
% \begin{itemize}
%     \item Odio l'ipocrisia e l'incoerenza.
% \end{itemize}

\cvsection{Pubblicazioni}
% %% Specify your last name(s) and first name(s) as given in the .bib to automatically bold your own name in the publications list.
% %% One caveat: You need to write \bibnamedelima where there's a space in your name for this to work properly; or write \bibnamedelimi if you use initials in the .bib
% %% You can specify multiple names, especially if you have changed your name or if you need to highlight multiple authors.
\mynames{Bressan\bibnamedelima Marco}
%   Wong/Lian\bibnamedelima Tze,
%   Lim/Tracy,
%   Lim/L.\bibnamedelimi T.}
% %% MAKE SURE THERE IS NO SPACE AFTER THE FINAL NAME IN YOUR \mynames LIST

\nocite{*}

% \printbibliography[heading=pubtype,title={\printinfo{\faBook}{Books}},type=book]

% \divider

\printbibliography[heading=pubtype,title={\printinfo{\faFile*[regular]}{Articoli su riviste}},type=article]

\printbibliography[heading=pubtype,title={\printinfo{\faFolderOpen[regular]}{Manuali e \textit{software}}},type=manual]

% \divider

%\printbibliography[heading=pubtype,title={\printinfo{\faUsers}{Conference Proceedings}},type=inproceedings]

%% Switch to the right column. This will now automatically move to the second
%% page if the content is too long.
\switchcolumn

% \cvsection{My Life Philosophy}

% \begin{quote}
% ``Something smart or heartfelt, preferably in one sentence.◔◑◕●''
% \end{quote}

% \cvsection{Most Proud of}

% \cvachievement{\faTrophy}{Fantastic Achievement}{and some details about it}

% \divider

% \cvachievement{\faHeartbeat}{Another achievement}{more details about it of course}

% \divider

% \cvachievement{\faHeartbeat}{Another achievement}{more details about it of course}

\cvsection{Competenze digitali}

% Don't overuse these \cvtag boxes — they're just eye-candies and not essential. If something doesn't fit on a single line, it probably works better as part of an itemized list (probably inlined itemized list), or just as a comma-separated list of strengths.

% The `ragged2e` document class option might cause automatic linebreaks between \cvtag to fail.
% Either remove the ragged2e option; or 
% add \LaTeXraggedright in the paragraph for these \cvtag

{\LaTeXraggedright\myskillsIT}

\medskip 

\cvsection{Lingue}

\textbf{Lingua madre:} Italiano

\divider

\textbf{Altre lingue:} \\[2mm]
Inglese \hfill \textbf{B2} - Utente autonomo \\[2mm]
Francese \hfill \textbf{A2} - Utente di base

%% Yeah I didn't spend too much time making all the
%% spacing consistent... sorry. Use \smallskip, \medskip,
%% \bigskip, \vspace etc to make adjustments.
\medskip

\cvsection{Formazione}

\myeducationIT

\medskip


\cvsection{Conferenze}

\myeventsIT

% \medskip

% % \divider

% \cvsection{Referees}

% % \cvref{name}{email}{mailing address}
% \cvref{Prof.\ Alpha Beta}{Institute}{a.beta@university.edu}
% {Address Line 1\\Address line 2}

% \divider

% \cvref{Prof.\ Gamma Delta}{Institute}{g.delta@university.edu}
% {Address Line 1\\Address line 2}


\end{paracol}

\vfill


% \footnotesize{
%     \color{white}
%     Data analysis
%     Machine learning
%     Data science
%     Artificial intelligence
%     Deep learning
%     Neural networks
%     Learning algorithms
%     Statistics
%     Probability
%     Data visualization
%     Data mining
%     Big data
%     NoSQL
%     SQL
%     Python
%     R
%     Julia
%     Scikit-learn
%     TensorFlow
%     Keras
%     PyTorch
%     Pandas
%     NumPy
%     Matplotlib
%     Seaborn
%     Plotly
%     Data preprocessing
%     Feature engineering
%     Model selection
%     Hyperparameter tuning
%     Cross-validation
%     Linear regression
%     Logistic regression
%     Classification
%     Clustering
%     Dimensionality reduction
%     Feature selection
%     Model evaluation
%     Optimization
%     Data science
%     Data engineering
%     Data architecture
%     Data management
%     Data quality
%     Data security
%     Data ethics
%     Data communication
%     Data storytelling
%     Data-driven decision making
%     Business intelligence
%     Data warehousing
%     ETL (Extract, Transform, Load)
%     Data governance
%     Data quality
%     Data security
%     Cloud computing
%     Azure
%     AWS
%     Google Cloud
%     Docker
%     Kubernetes
%     Analisi dati
%     Machine learning
%     Data science
%     Intelligenza artificiale
%     Deep learning
%     Reti neurali
%     Algoritmi di apprendimento
%     Statistica
%     Probabilità
%     Visualizzazione dati
%     Data mining
%     Big data
%     NoSQL
%     SQL
%     Python
%     R
%     Julia
%     Scikit-learn
%     TensorFlow
%     Keras
%     PyTorch
%     Pandas
%     NumPy
%     Matplotlib
%     Seaborn
%     Plotly
%     Data preprocessing
%     Feature engineering
%     Model selection
%     Hyperparameter tuning
%     Cross-validation
%     Regressione lineare
%     Regressione logistica
%     Classificazione
%     Clustering
%     Riduzione della dimensionalità
%     Selezione delle caratteristiche
%     Valutazione dei modelli
%     Ottimizzazione
%     Scienza dei dati
%     Ingegneria dei dati
%     Architettura dei dati
%     Gestione dei dati
%     Qualità dei dati
%     Sicurezza dei dati
%     Etica dei dati
%     Comunicazione dei dati
%     Storytelling con i dati
%     Data-driven decision making
%     Business intelligence
%     Data warehousing
%     ETL (Extract, Transform, Load)
%     Data governance
%     Data quality
%     Data security
%     Cloud computing
%     Azure
%     AWS
%     Google Cloud
%     Docker
%     Kubernetes
% }

\end{document}
